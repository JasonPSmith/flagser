\documentclass{amsart}

\usepackage[utf8]{inputenc}

\usepackage[]{amsmath}
\usepackage{amsthm}
\usepackage{latexsym}
\usepackage{amssymb}
\usepackage{mathrsfs}
\usepackage{exscale}
\usepackage{textcomp}
\usepackage[all,ps,tips,tpic]{xy}
\usepackage{upgreek}
\usepackage{url}
\usepackage{booktabs}
\usepackage[final,pdftex,colorlinks=false,pdfborder={0 0 0}]{hyperref}
\usepackage{microtype}
\usepackage{verbatim}
\usepackage{soul}

\theoremstyle{definition}
\newtheorem*{example*}{Example}

%spacing around left-right brackets
\let\originalleft\left
\let\originalright\right
\renewcommand{\left}{\mathopen{}\mathclose\bgroup\originalleft}
\renewcommand{\right}{\aftergroup\egroup\originalright}

\begin{document}
\title{Documentation: flagser}

\author{Daniel Lütgehetmann}
\address{University of Aberdeen, Aberdeen, United Kingdom}
\email{daniel.lutgehetmann@abdn.ac.uk}
\urladdr{https://homepages.abdn.ac.uk/daniel.lutgehetmann/pages/}

\maketitle

\noindent
The software \textsc{flagser} is an adaptation of Ulrich Bauer's
\href{https://github.com/Ripser/ripser}{\textsc{ripser}}, enabling the computation of persistent
homology for directed flag complexes with different filtrations.

\section{Requirements}
\noindent
\textsc{flagser} requires a C++14 compiler and \textsc{CMake}.
For custom filtration algorithms, additionally \textsc{python} (including $\textsc{pip}$) is required.
To see the \textsc{python} packages needed in this case, see \texttt{requirements.txt}.
If you want to read .h5 files, you need to install the HDF5 library as well (https://support.hdfgroup.org/HDF5/).

\section{Compiling the source code}
\noindent
Open the command line, change into the \textsc{flagser} main directoy and execute

\vspace{1em}

\begin{verbatim}(mkdir -p build && cd build && cmake .. && make -j)\end{verbatim}

\vspace{1em}

\noindent
\textit{Specific for MacOS:} on the first run, you may be asked to install the XCode Developer Tools, which you should confirm.
After the automatic installation finished, just run the command above again.
For the preinstalled \textsc{python} there might be problems installing the necessary libraries.
If the compilation is not successful, execute \texttt{sudo easy\_install pip} and try again.

\section{Usage}
\noindent
After building \textsc{flagser}, you can compute persistent homology as follows:

\vspace{1em}

\begin{verbatim}./flagser [options] filename\end{verbatim}

\vspace{1em}

You can increase the speed of computation by using more memory:

\vspace{1em}

\begin{verbatim}./flagser-memory [options] filename\end{verbatim}

\vspace{1em}

\noindent
The following \texttt{[options]} exist:

\enlargethispage{\baselineskip}
\begin{description}
  \item [-{}-out \textit{filename}] write the barcodes to the given file
  \item [-{}-out-format \textit{format}] the output format, which is either \texttt{barcodes} or
    \texttt{betti}. The default is \texttt{barcodes}
  \item [-{}-in-format \textit{format}] the input format, which is either \texttt{h5} for HDF5 files
    or \texttt{flagser} for a \textsc{flagser} file (see below). For the h5 input format the HDF5
    library needs to be installed before building the source code. The format defaults to
    \texttt{flagser}
  \item [-{}-h5-type \textit{type}] the type of data in the h5-file. The type can either be
    \texttt{"matrix"} if at the given path in the HDF5-file there is the connectivity matrix or \texttt{"grouped"}
    if the connectivity matrices are grouped. To only consider a subset of the groups you can list
    them after \texttt{"grouped"}, e.g. \texttt{"grouped:L1\_DAC,L2*"}. The star is a placeholder for
    arbitrary characters. The type defaults to \texttt{"matrix"} and is only relevant for the input
    type \texttt{h5}
  \item [-{}-filtration \textit{algorithm}] use the specified algorithm to compute the filtration. \textbf{Warning}:
  if an edge filtration is specified, it is assumed that the resulting filtration is consistent, meaning that the
  filtration value of every simplex of dimension at least two should evaluate to a value that is at least the
  maximal value of the filtration values of its containing edges. For performance reasons, this is not checked
  automatically.
  \item [-{}-max-dim \textit{dim}] the maximal homology dimension to be computed
  \item [-{}-min-dim \textit{dim}] the minimal homology dimension to be computed
  \item [-{}-approximate \textit{n}] skip all cells creating columns in the reduction matrix with
    more than $n$ entries. Use this for hard problems, a good value is often 100000. Increase for
    higher precision, decrease for faster computation. Warning: for non-trivial filtrations, this
    makes for hard to estimate theoretical bounds on the error, although usually the real error is
    much lower than the theoretical one. Use for exploration and validate later with longer computation
    time
  \item [-{}-components] compute the directed flag complex for each individual connected
    component of the input graph. \emph{Warning: this currently only works for the trivial
    filtration. Additionally, this ignores all isolated vertices.}
  \item [-{}-undirected] computes the \emph{undirected} flag complex instead
  \item [-{}-help] print a help screen
\end{description}
The available filtration algorithms are printed when executing \texttt{./flagser --help}.
For the input format \texttt{h5}, you can select the path inside the h5-file by appending it to the
filename, e.g.\ \texttt{./flagser \ldots filename.h5/experiment1/connectivity}.

\vspace{1em}

\noindent
The input file defines the directed graph and must have the following shape (if the option
\texttt{-{}-h5} is not used):

\vspace{.5em}
\begin{verbatim}
dim 0:
weight_vertex_0 weight_vertex_1 ...  weight_vertex_n
dim 1:
first_vertex_id_of_edge_0 second_vertex_id_of_edge_0 [weight_edge_0]
first_vertex_id_of_edge_1 second_vertex_id_of_edge_1 [weight_edge_1]
...
first_vertex_id_of_edge_m second_vertex_id_of_edge_m [weight_edge_m]
\end{verbatim}
\vspace{.5em}

\noindent
The edges are oriented to point from the first vertex to the second vertex, the weights of the edges
are optional.
The weights should be rational numbers.

\begin{example*}
  The full directed graph on three vertices without explicit edge weights is described by the
  following input file:

  \vspace{.5em}
  \begin{verbatim}
  dim 0:
  0.2 0.522 4.9
  dim 1:
  0 1
  1 0
  0 2
  2 0
  1 2
  2 1
  \end{verbatim}
\end{example*}

\section{Tools}
You can use the following tools to generate input matrices for \textsc{flagser}.

\subsection{H5 to \textsc{flagser}}
To convert a connection matrix in the H5-format to an input file for \textsc{flagser}, call

\noindent
\verb| |\texttt{\qquad./tools/h5toflagser [--groups \textit{"GROUP1,GROUP2,\dots"}]} \\
\verb| |\texttt{\qquad\qquad[--random-edge-filtration] \textit{h5file} \textit{output\_filename}}

With the parameter \texttt{groups} you can extract the subgraph consisting only of neurons in the
given groups, otherwise the full graph is constructed. You can use a star in the group names to
match multiple groups, e.g.\ \texttt{--groups "L5\_*"} matches all groups in layer 5.
If you pass the parameter \texttt{--random-edge-filtration}, then each edge is assigned a random
weight in $[0,1]$.

\begin{example*}
  To construct the full second column of the average rat, run

  \begin{verbatim}
    ./tools/h5toflagser cons_locs_pathways_mc2_Column.h5
          cons_locs_pathways_mc2_Column.flag
  \end{verbatim}

  To only extract layer 4 and layer 5, run

  \begin{verbatim}
    ./tools/h5toflagser --groups "L4_*,L5_*"
          cons_locs_pathways_mc2_Column.h5
          cons_locs_pathways_mc2_Column.flag
  \end{verbatim}
\end{example*}

\subsection{Erdős–Rényi graphs}
To construct an Erdős–Rényi graph on $n$ vertices with density $d\in[0,1]$, call

\noindent
\verb| |\qquad\texttt{./tools/er
[--density \textit{d}] [--random-edge-filtration] \textit{n} \textit{output\_filename}}

\noindent
If the density is not specified, it is assumed to be $d=1$.
If you pass the parameter \texttt{--random-edge-filtration}, then each edge is assigned a random
weight in $[0,1]$.

\begin{example*}
  Calling \texttt{./tools/er --density 0.314 20 er-20.flag} creates the \textsc{flagser} input file
  for a graph on 20 vertices where $31.4\%$ of all \emph{directed} edges are present.
\end{example*}

\subsection{Counting flag complex cells}
To just count the number of cells of the directed flag complex, call

\noindent
\verb| |\qquad\texttt{./flagser-count [options] \textit{input\_filename}}
You can use the same input options as for \texttt{flagser}. If you specify an output file by

\noindent
\verb| |\qquad\texttt{./flagser-count --out filename.h5 \ldots}

\noindent
then a list of cells is saved into an existing or new HDF5-file. If you specify \texttt{--out
filename.h5/some/path}, then the data will be written into the subgroup \texttt{/some/path}.

\section{Writing custom filtration algorithms}
The filtration of the directed flag complex is computed by algorithms.
Examples of such algorithms are \texttt{max} (assigning each cell the maximal weight of its
boundary cells) and zero (assigning each cell the weight zero).
To implement a custom algorithm you extend the file \texttt{algorithms.math} by another definition.
This definition needs to be written in a special tiny programming language resembling mathematical
notation.
It is described in detail below, but here is an example:

\begin{example*}
The code
\begin{verbatim}
  square_mean = max(faceWeights)
    + sum(faceWeights, x -> x^2)^(1/2) / (dimension + 1)
\end{verbatim}
assigns each cell $\sigma$ the weight
\[
  f(\sigma) = \max_{\nu\in\partial(\sigma)} f(\nu) +
  \left(\Sigma_{\nu\in\partial(\sigma)} f(\nu)^2\right)^{1/2}\cdot
  \frac{1}{\dim(\sigma) + 1}
\]
You can find more examples by looking at the built-in functions in \texttt{algorithms.math}.
\end{example*}

\subsection{Defining a filtration algorithm}
To add a new filtration algorithm, think of a name (consisting only of the letters a to z and
underscores) and add a new line of the form
\begin{verbatim}
your_chosen_name [overrideVertices] [overrideEdges]
    = formula_to_compute_the_weight
\end{verbatim}
into \texttt{algorithms.math}, where \texttt{formula\_to\_compute\_the\_weight} is an expression
as described in the next section.
The optional keywords \texttt{overrideVertices} and \texttt{overrideEdges} indicate that the
weights given to the vertices and edges in the input file are replaced by values computed
via this algorithm.
For example, the trivial filtration algorithm reads as follows:
\begin{verbatim}
zero overrideVertices overrideEdges = 0
\end{verbatim}
You have to recompile \textsc{flagser} by executing \texttt{make} again and can then use the new
filtration algorithm by passing the option \texttt{--filtration your\_chosen\_name} to
\textsc{flagser}.

\subsection{The mathematical input language}
The language for defining filtration algorithms is given in a special format, described in this
section.
You can use basic arithmetic expressions \verb|+, -, *, /, ^| and the exponential map \verb|exp( )|.
In addition to that, you can access several constants like the dimension of the current cell or
weights, as well as built-in functions.
Below we give a list of all these available constants and commands.

\vspace{1em}

On the top level you can perform a case distinction by dimension:

\noindent
\textbf{cases: dim \textit{n1}: \textit{formula1} dim \textit{n2}: \textit{formula2} \ldots else:
\textit{formula}}
\begin{example*}
  To assign vertices and edges the value 0 and all other faces their dimension as weights,
  we would define
  \begin{verbatim}
  modified_dimension = cases:
    dim 0: 0
    dim 1: 0
    else: dimension
  \end{verbatim}
\end{example*}
If an algorithm requires a certain filtration to be provided by the input file (which is only
reasonable for dimension 0 and 1 of course), then you can add an error message shown to the user
whenever this data is missing by using \textbf{error~\textit{error\_message}}
\begin{example*}
  If you require weights for all vertices and edges, you write a filtration algorithm as
  follows:
  \begin{verbatim}
  needs_vertex_and_edge_weights = cases:
    dim 0: error "Please provide weights for the vertices."
    dim 1: error "Please provide weights for the edges."
    else: some_formula
  \end{verbatim}
\end{example*}

As available constants for your formulas you can use the following:
\begin{description}
  \item[dimension] the dimension of the current cell
  \item[faceWeights] the list of weights of the faces of the current cell
  \item[cellVertices] the ordered list of vertices of the current cell
  \item[vertexOutDegrees] the outgoing degrees of all vertices of the graph
  \item[vertexInDegrees] the incoming degrees of all vertices of the graph
  \item[vertexWeights] the weights of all vertices of the graph
\end{description}
\begin{example*}
  You get individual entries of the lists above by appending \texttt{[index]} to them, e.g.
  \begin{verbatim}  cellVertices[0]\end{verbatim}
  gives the first vertex in the cell,
  \begin{verbatim}  faceWeights[dimension]\end{verbatim}
  gives the weight of the last face of the current cell and
  \begin{verbatim}  vertexInDegrees[cellVertices[2]]\end{verbatim}
  gives the incoming degree of the third (notice the indexing starting from zero) vertex of the
  current cell.
\end{example*}

For the lists (except for the vertex degrees and the vertex weight) you can use the following
functions:

\noindent
\textbf{min/max(\textit{list} [, \textit{from}, \textit{to}])}:
These commands return the minimal and maximal element of the given list.
If you pass \texttt{from} and \texttt{to}, the elements before \texttt{from} and after \texttt{to}
are disregarded.
If the specified part of the list is empty, then 0 is returned.
\begin{example*}
  The following examples show the usage of those two functions.
  \begin{verbatim}
  maximum = max(faceWeights)
  maximum_plus_minimum_of_first_three_faces
      = max(faceWeights) + min(faceWeights, 0, 2)
  \end{verbatim}
\end{example*}


\noindent
\textbf{sum/product(\textit{list} [, \textit{from}, \textit{to}] [, \textit{function}])}:
These commands return the sum or the product of the elements in the given list.
If you pass \texttt{from} and \texttt{to}, the elements before \texttt{from} and after \texttt{to}
are disregarded.
If the specified part of the list is empty, then 0 is returned.

If you pass a \texttt{function}, each element is mapped through this function before adding or
multiplying.
A \textbf{function} is defined in the form \verb+x -> expression_in_x+, for example
\begin{verbatim}
    x -> x^2
\end{verbatim}
denotes the function that squares each element,
\begin{verbatim}
    z -> 3*z-5*exp(1/z)
\end{verbatim}
denotes the function $z\mapsto 3z-5e^{1/z}$.
\begin{example*}
  The following examples show the usage of those two functions.
  \begin{verbatim}
  sum = sum(faceWeights)
  sum_squared = sum(faceWeights, 2, dimension, x -> x^2)
  product_vertex_weights
      = product(cellVertices, v -> vertexWeights[v])
  \end{verbatim}
\end{example*}

\noindent
\textbf{combine/reduce(\textit{list} [, \textit{from}, \textit{to}], \textit{function},
\textit{initial\_value})}:
These two commands are the same, and they combine elements in a list into a single number.
If you pass \texttt{from} and \texttt{to}, the elements before \texttt{from} and after \texttt{to}
are disregarded.
If the specified part of the list is empty, then \texttt{initial\_value} is returned.

The \texttt{function} you pass must take two elements and produce a third one, i.e.\ it must be of
the form \verb+carry x -> expression_in_carry_and_x+.
The first argument of the function is the accumulated result, the second one is the next value.
In the first step, \texttt{initial\_value} is passed as the accumulated result.

For example, we have the following equivalences:
\begin{verbatim}
  sum(array) == combine(array, carry x -> carry + x, 0)
  product(array) == combine(array, carry x -> carry * x, 1)
\end{verbatim}

\begin{example*}
  The following example shows the usage of this function.
  \begin{verbatim}
  squaring = combine(
    faceWeights,
    carry x -> (carry + x)^2,
    vertexWeights[cellVertices[0]]
  )
  \end{verbatim}
\end{example*}

\vfill

\end{document}
